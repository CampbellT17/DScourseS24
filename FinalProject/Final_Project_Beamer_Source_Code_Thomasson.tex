\documentclass{beamer}
\usetheme{default}

% Title slide
\title{Are Earnings with Large Book-Tax Differences Reliable? A Replication and Extension of Hanlon (2005)}
\author{Campbell Thomasson}
\date{\today}

\begin{document}

\begin{frame}
  \titlepage
\end{frame}

% Introduction
\begin{frame}{Introduction}
\begin{itemize}
    \item  Companies use different rules of reporting income. Financial Statements use GAAP. Tax Returns use the IRC.
    \item  Differences are known as Book-Tax Differences. Typically, these are only timing differences and will resolve over time.  
    \item  However, companies are incentivized to report high book income and low taxable income, so if the book-tax differences are large, it may be a sign of earnings management and/or tax avoidance.
\end{itemize}
\end{frame}

% Hypotheses
\begin{frame}{Hypotheses}
\begin{itemize}
    \item \textbf{H1:} Pre-tax earnings persistence for firm-years with large negative or large positive book-tax differences is lower than pre-tax earnings persistence for firm-years with small book-tax differences.
    \item  \textbf{H2:} The persistence of the accruals component of earnings for future earnings is lower for firm-years with large negative or large positive book-tax differences relative to firm-years with small book-tax differences.
\end{itemize}
\end{frame}

% Data
\begin{frame}{Data}
\begin{itemize}
    \item  All data was collected from Compustat via the SAS Studio server provided by Wharton.
    \item  Deferred Tax Expense = Book-Tax-Differences. 
    \item  PTBI = Pre-Tax Book Income
    \item  PTCF = Pre-Tax Cash Flows
    \item  PTACC = Pre-Tax Accruals
    \item  LBTD = Large Book-Tax Differences (Top or Bottom Quintile)
    \item  LPBTD = Large Positive Book-Tax Differences (Book > Tax) (Top Quintile)
    \item  LNBTD = Large Negative Book-Tax Differences (Book < Tax) (Bottom Quintile)
\end{itemize}
\end{frame}

% Methods
\begin{frame}{Methods - PTBI}
    \item $PTBI_{t+1} = \beta_0 + \beta_1PTBI_t + \varepsilon_{t+1}$

    \vspace{0.5cm}
    
    \item $PTBI_{t+1} = \beta_0 + \beta_1LBTD_t + \beta_2PTBI_t + \beta_3PTBI_t * LBTD_t + \varepsilon_{t+1}$

    \vspace{0.5cm}
    
    \item $PTBI_{t+1} = \beta_0 + \beta_1LNBTD_t + \beta_2LPBTD_t + \beta_3PTBI_t + \beta_4PTBI_t * LNBTD_t + \beta_5PTBI_t * LPBTD_t + \varepsilon_{t+1}$
\end{frame}

% Methods - PTBI Components
\begin{frame}{Methods - PTBI Components}
    \item $(4) PTBI_{t+1} = \beta_0 + \beta_1PTCF_t + \beta_2PTACC_t + \varepsilon_{t+1}$

    \vspace{0.5cm}
    
    \item $(5) PTBI_{t+1} = \beta_0 + \beta_1LBTD_t + \beta_2PTCF_t + \beta_3PTACC_t + \beta_4PTCF_t * LBTD_t + \beta_5PTACC_t * LBTD_t + \varepsilon_{t+1}$

    \vspace{0.5cm}
    
    \item $(6) PTBI_{t+1} = \beta_0 + \beta_1LNBTD_t + \beta_2LPBTD_t + \beta_3PTCF_t + \beta_4PTACC_t + \beta_5PTCF_t * LNBTD_t + \beta_6PTCF_t * LPBTD_t + \beta_7PTACC_t * LNBTD_t + \beta_8PTACC_t * LPBTD_t + \varepsilon_{t+1}$
\end{frame}

% Findings
\begin{frame}{Findings}
  \begin{table}[htbp]
\centering
\caption{PTBI Persistence Results}
\label{tab:Models_1_3_Results}
\begin{tabular}{|c|c|c|c|}
\hline
Variable & (1) & (2) & (3) \\
\hline
Intercept & 0.025*** & 0.019*** & 0.019*** \\
scaled\_pi & 0.620*** & 0.681*** & 0.681*** \\
lbtd & & 0.0141*** & \\
pi\_lbtd &  & -0.123*** & \\
lnbtd &  &  & 0.020*** \\
lpbtd &  &  & 0.009*** \\
pi\_lnbtd &  &  & -0.121*** \\
pi\_lpbtd &  &  & -0.130*** \\
\hline
Adj $R^2$ & 0.495 & 0.500 & 0.502 \\
\hline
N & 24,931 & 24,931 & 24,931 \\
\hline
\end{tabular}
\end{table}
\end{frame}

% Findings
\begin{frame}{Findings}
  \begin{table}[htbp]
\centering
\caption{Pre-Tax Earnings Components Results}
\label{tab:Models_4_6_Results}
\begin{tabular}{|c|c|c|c|}
\hline
Variable & (4) & (5) & (6) \\
\hline
Intercept & 0.020*** & 0.015*** & 0.015*** \\
scaled\_ptcf & 0.634*** & 0.692*** & 0.692*** \\
scaled\_ptacc & 0.480*** & 0.556*** & 0.556*** \\
lbtd &  & 0.010*** & \\
ptcf\_lbtd &  & -0.114*** &  \\
ptacc\_lbtd &  & -0.150*** &  \\
lnbtd &  &  & 0.016*** \\
lpbtd &  &  & 0.005* \\
ptcf\_lnbtd &  &  & -0.117*** \\
ptcf\_lpbtd &  &  & -0.113*** \\
ptacc\_lnbtd &  &  & -0.150*** \\
ptacc\_lpbtd &  &  & -0.145*** \\
\hline
Adj $R^2$ & 0.503 & 0.508 & 0.510 \\
\hline
N & 24,931 & 24,931 & 24,931 \\
\hline
\end{tabular}
\end{table}
\end{frame}

%How Did AI Help?
\begin{frame}{How did AI Help?}
\begin{itemize}
    \item  ChatGPT made this slide template and showed me how to actually make a Beamer.
    \item  ChatGPT also took my research paper from my Microsoft Word Document and wrote the whole thing in LaTeX script for me.
\end{itemize}
\end{frame}

% Conclusion
\begin{frame}{Conclusion}
\begin{itemize}
    \item  Large Book-Tax Differences are associated with less persistent, and therefore lower quality, reported earnings.
    \item  Cash Flows are a better predictor of future performance than accruals, but both components are less predictive in firms with large book-tax differences.
    \item  Whatever choices led you to not becoming an accounting researcher - you can still feel good about those choices.
\end{itemize}
\end{frame}

\end{document}
